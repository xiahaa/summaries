\documentclass[a4paper]{report}
\usepackage{lipsum}
\usepackage{tikzpagenodes}
\usepackage{pgfplots}
\usepackage{tikz}
\usepackage{tikz-3dplot}
\usetikzlibrary{arrows,decorations.pathmorphing,backgrounds,positioning,fit,matrix}
\pgfplotsset{compat=1.8}
\usepackage{graphics} % for pdf, bitmapped graphics files
\usepackage{epsfig} % for postscript graphics files
\usepackage[colorlinks=true,citecolor=green]{hyperref}
\usepackage{cite}
\usepackage{amsmath,amssymb,amsfonts}
\usepackage{algorithmic}
\usepackage{graphicx}
\usepackage{url}
\usepackage{cite}
\usepackage{bm}
\usepackage{pbox}
\usepackage{siunitx,booktabs,etoolbox}

\def\BibTeX{{\rm B\kern-.05em{\sc i\kern-.025em b}\kern-.08em
    T\kern-.1667em\lower.7ex\hbox{E}\kern-.125emX}}

\begin{document}

\section{DLT initialization}
Assume the state to estimate is $\mathbf{x} = [x,\ y,\ z,\ dT]^T$. To use DLT for initial estimation, we have to assume $dT = 0$, then we will have the following relationship between the pseudorange and the geometric distance:
\begin{align}
\rho &= \sqrt{(x-x_i)^2+(y-y_i)^2+(z-z_i)^2},\ i=1,2,\cdots,n \\
{\rho}^2 &= {(x-x_i)^2+(y-y_i)^2+(z-z_i)^2},\ i=1,2,\cdots,n \\
d\rho_{ij}^2 &= {\rho}^2_i - {\rho}^2_j \\
d\rho_{ij}^2 &={(x-x_i)^2+(y-y_i)^2+(z-z_i)^2}-{(x-x_j)^2-(y-y_j)^2-(z-z_j)^2}
\end{align}
Do some simple extensions, we will have:
\begin{align}
d\rho_{ij}^2 - (x_i^2+y_i^2+z_i^2) + (x_j^2+y_j^2+z_j^2) &= 2(x_j-x_i)x \nonumber \\
& + 2(y_j-y_i)y + 2(z_j-z_i)z \nonumber \\
&=\left[ 
\begin{matrix}
2(x_j-x_i) & 2(y_j-y_i) & 2(z_j-z_i) \\
\end{matrix}
\right]
\left[ 
\begin{matrix}
x \\ y \\ z \\
\end{matrix}
\right]
\end{align}
By stacking all measurements together, we will have a linear equation $\mathbf{A}[x,\ y,\ z]^T=\mathbf{b}$. By solving this linear system, we will have an estimation of $\mathbf{x} = [x,\ y,\ z,\ dT]^T = [x_0,\ y_0,\ z_0,\ 0]^T$. This simple DLT initialization will make the non-linear optimization procedure converge more quickly. Moreover, since it can work will any prior guess of $\mathbf{x}$ and given a better guess, it will not influence the final result, i.e. the local minimum for initial guess and that of the DLT guess are the same.

\section{SOCP initialization}
The initialization can also be obtained with the Second-Order-Cone Programming (SOCP). Revisit the objective function:
\begin{equation}
\text{minimize}:\ \sum\limits_{i=0}^{n}{\left(\rho_i - (\sqrt{(x-x_i)^2+(y-y_i)^2+(z-z_i)^2}+cdT)\right)^2} 
\end{equation}
Now, use some slack variables $l_i=\sqrt{(x-x_i)^2+(y-y_i)^2+(z-z_i)^2}$ and the epigraph $t \geq \sum\limits_{i=0}^{n}{\rho_i - l_i - cdT)^2}$, we will have:
\begin{align}
\text{minimize}&:\ t \nonumber \\
\text{subject to}&: t \geq \sum\limits_{i=0}^{n}{\rho_i - l_i - cdT)^2} \\
& l_i=\sqrt{(x-x_i)^2+(y-y_i)^2+(z-z_i)^2}, \ i=1,2,\cdots,n \nonumber 
\end{align}
Relax  $l_i=\sqrt{(x-x_i)^2+(y-y_i)^2+(z-z_i)^2}$  to  $l_i\geq\sqrt{(x-x_i)^2+(y-y_i)^2+(z-z_i)^2}$, we will have the standard SOCP problem:
\begin{align}
\text{minimize}&:\ t \nonumber \\
\text{subject to}&: 
(t,\frac{1}{1},\rho_0 - l_0 - cdT,\cdots,\rho_n - l_n - cdT) \in \mathcal{Q}_r^{n+3} \\
& (l_i,(x-x_i),(y-y_i),(z-z_i)) \in \mathcal{Q}^{4}, \ i=1,2,\cdots,n \nonumber 
\end{align}
By solving this convex SOCP problem, we will have an good estimation as $\mathbf{x} = [x,\ y,\ z,\ dT]^T = [x_0,\ y_0,\ z_0,\ cdT_0]^T$. The relaxation can be controlled by adding regularizations as:
\begin{align}
\text{minimize}&:\ t+\lambda\sum\limits_{i=0}^{n}{l_i} \nonumber \\
\text{subject to}&: 
(t,\frac{1}{1},\rho_0 - l_0 - cdT,\cdots,\rho_n - l_n - cdT) \in \mathcal{Q}_r^{n+3} \\
& (l_i,(x-x_i),(y-y_i),(z-z_i)) \in \mathcal{Q}^{4}, \ i=1,2,\cdots,n \nonumber 
\end{align}
where $\lambda$ can be seen as a trade-off weight.
\section{Augmented State}
Revisit the composition of the pseudorange (omitting the satellite orbit error):
\begin{align}
\rho &= R + \underbrace{cdT}_{receiver} - \underbrace{cdt}_{satellite} + \underbrace{d_{iono}}_{satellite, receiver} + \underbrace{d_{trop}}_{satellite, receiver} \\
R &= \sqrt{(x-x_{sv})^2+(y-y_{sv})^2+(z-z_{sv})^2}
\end{align}
Among those errors, 
\begin{itemize}
\item $cdT$ is receiver-dependent, identical for pseudorange measurements;
\item $cdt$ is satellite-dependent, unique for each pseudorange measurement;
\item $d_{iono}$ and $d_{trop}$,are satellite and receiver dependent implicitly;
\end{itemize}
A further analysis can be concluded as follows:
\begin{itemize}
\item Since $cdt$ is hard to model (in fact we need to add one unknown parameter for each pseudorange which means the number of unknown parameters are more than the number of measurements), it should be definitely eliminated before carrying out further computation.
\item $cdT$ can be modelled as one parameter to resolve. This is what we do normally.
\item According to the common used model for modelling the ionospheric and tropospheric errors, $d_{iono}$ and $d_{trop}$ are related to their corresponding errors in polar direction (which is receiver and satellite independent) and the corresponding elevation angles. Consequently, we can add two more parameters $x_5$, $x_6$ and model the ionospheric error and tropospheric error as:
\begin{align}
d_{iono}&= x_5 * OF,\ OF = \left(1-\left(\frac{RE*sin(zenith)}{RE+hI}\right)^2\right)^{-1/2} \\
d_{trop}&= \frac{x_6}{sin(elv)}
\label{eq:ionotrop}
\end{align}
With known satellite positions, Eq~\ref{eq:ionotrop} can be written implicitly as the function of $\mathbf{x}={x,\ y,\ z,\ cdT}$ as
\begin{align}
d_{iono}&= x_5 * f_1(\mathbf{x}, x_{sv}) \\
d_{trop}&= \frac{x_6}{ f_2(\mathbf{x}, x_{sv})} \\
\Leftrightarrow \nonumber \\
d_{iono}&= f_1^*(\mathbf{x}_{aug}) \\
d_{trop}&= f_2^*(\mathbf{x}_{aug}) \\
\mathbf{x}_{aug} &= [x,\ y,\ z,\ cdT,\ x_5,\ x_6]^T \nonumber
\end{align}
In this way, there is no need to eliminate the ionospheric delay and the tropospheric delay from the pseudoranges. Instead, we estimate $x_5,\ x_6$ together with $x,\ y,\ z,\ cdT$ through least-square optimization. The benefit of doing this is that we don't need to estimate the ionospheric delay and the tropospheric delay which depend on various factors and therefore hard to model. The cost we need to pay for this state augmentation is that the minimum number of visible satellites increases to 6 since there are 6 unknown parameters. Since now there the GPS, GLONASS, Galileo, Beidou system operated, observing at least 6 satellites are not so hard in open sky environment.
\end{itemize}
\textbf{navSolverAug.m}: do position estimation using the augmented state vector.

%\pbox[c][20pt][b]{\textwidth}{Estimation\footnote{Pseudoranges are corrected with satellite clock error.}\\ \vphantom{g}}

\subsection{Result}
\begin{table}[h]
\centering
\begin{tabular}{c|c|c|c|c}
\hline
\textbf{} & \textbf{x}: (m) & \textbf{y}: (m)  & \textbf{z}: (m)  & \textbf{dT}: (s) \\ \hline 
truth & 3509042.2969  &    779567.15431  & 5251066.1743 & 0.0001 \\ \hline
Estimation 0 & 3380435.0196    &  721266.27907 &     5082767.4168 & -0.00028038122 \\ \hline  
Estimation 1 & 3509053.135     & 779569.77236  &    5251078.7139 & 0.00010006084 \\ \hline 
Estimation 2 & 3509042.2969     & 779567.15431  &    5251066.1743 & 0.0001 \\ \hline 
Estimation 3 & 3509042.2969     & 779567.15431  &    5251066.1743 & 0.0001  \\ \hline
\end{tabular}
\end{table}
\begin{itemize}
\item Estimation 0: Raw pseudoranges with no corrections;
\item Estimation 1: Pseudoranges are corrected with satellite clock error, normal state vector $\mathbf{x}_{aug} = [x,\ y,\ z,\ cdT]^T$;
\item Estimation 2: Pseudoranges are corrected with satellite clock error, augmented state vector $\mathbf{x}_{aug} = [x,\ y,\ z,\ cdT,\ x_5,\ x_6]^T$;
\item Estimation 3: Pseudoranges are corrected with satellite clock error, the ionosperic delay and the tropospheric delay, normal state vector $\mathbf{x}_{aug} = [x,\ y,\ z,\ cdT]^T$;
\end{itemize}

\begin{table}[h]
\centering
\begin{tabular}{
  l|
  c<{}@{  }|
  c<{}@{  }|
}
\toprule
 & $d_{iono}$: (m) & $d_{trop}$: (m) \\
\midrule
Truth & $2.009719031271335$ &   $3.072991650267203$ \\ 
Estimation & $2.060316692498627$  &  $3.165240666428296$  \\ \hline
Truth & $4.099930073823621$ &   $9.693598483115005$ \\ 
   Estimation &    $4.189709480973081$  &  $10.333536458801328$  \\ \hline
   Truth & $2.116069134646013$ &   $3.268438024695639$ \\ 
   Estimation &    $2.096887113400555$  &  $ 3.232743676464965 $ \\ \hline
   Truth & $2.839297088726820$ &   $4.803878722772577$ \\ 
   Estimation &    $2.766711098092753$  &  $ 4.628835520059784 $ \\ \hline
   Truth & $1.771088164677799$ &   $2.654116559350425$ \\ 
   Estimation &    $1.796907826493772$  &  $ 2.698253195570821 $ \\ \hline
   Truth & $2.501404314462073$ &   $4.034040635214109$ \\ 
   Estimation &    $2.412980142617273$  &  $ 3.849286796828415 $ \\ \hline
   Truth & $1.644024499990154$ &   $2.440639842034833$ \\ 
   Estimation &    $1.652890921255600$  &  $ 2.455344001554867 $ \\ \hline
   Truth & $2.480817599439596$ &   $3.990483231007696$ \\ 
   Estimation &    $2.518604983197980$  &  $ 4.070696026094213 $ \\ \hline
   Truth & $4.683754370548995$ &   $16.666993910233707$ \\ 
   Estimation &    $4.697532505190646$  &  $ 16.989111027376644$  \\ \hline
   Truth & $3.621709536037996$ &   $7.231334175028028$ \\ 
   Estimation &    $3.498462352983300$  &  $ 6.758697779975225 $ \\ \hline
   Truth & $3.025565441330769$ &   $5.282493672102815$ \\ 
   Estimation &    $3.173145222014883$  &  $ 5.697095454240499 $ \\ \hline
   Truth & $2.218178445425964$ &   $3.461948842712789$ \\ 
   Estimation &    $2.155787874337332$  &  $ 3.342995980093863 $ \\ \hline
\bottomrule
\end{tabular}
\caption{Comparison of true and estimated ionospheric and tropospheric delay. }
\label{tab:TAB-TAB}
\end{table}
As can be seen, the estimations of the ionospheric delay and the tropospheric delay are fairly good with the mean errors being $-6.982093895557447e-04$ m and $-0.051739902912841$ m and the standard deviation being $0.078216111070952$ m and $0.297526445119276$ m.

\end{document}