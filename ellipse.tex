\documentclass[a4paper]{report}
\usepackage{lipsum}
\usepackage{tikzpagenodes}
\usepackage{pgfplots}
\usepackage{tikz}
\usepackage{tikz-3dplot}
\usetikzlibrary{arrows,decorations.pathmorphing,backgrounds,positioning,fit,matrix}
\pgfplotsset{compat=1.8}
\usepackage{graphics} % for pdf, bitmapped graphics files
\usepackage{epsfig} % for postscript graphics files
\usepackage[colorlinks=true,citecolor=green]{hyperref}
\usepackage{cite}
\usepackage{amsmath,amssymb,amsfonts}
\usepackage{algorithmic}
\usepackage{graphicx}
\usepackage{url}
\usepackage{cite}
\usepackage{bm}
\usepackage{pbox}
\usepackage{siunitx,booktabs,etoolbox}

\def\BibTeX{{\rm B\kern-.05em{\sc i\kern-.025em b}\kern-.08em
    T\kern-.1667em\lower.7ex\hbox{E}\kern-.125emX}}

\begin{document}

\section{Introduction}


%%% dong ntu 
An ellipse (or a circle) is a common man-made nonlinear geo- metric shape. 

The ellipse detection from image streams is a common need in a variety of applications such as industrial inspection [1] , medical diagnosis [2,3] , recognition of traffic signs [4] , security [5] , face recognition [6,7] , and object tracking for a robotic plat- form [8–10] . Thus

reference no use

The presence of noise substantially overwhelms edge pixels of real ellipses and breaks an ellipse’s boundary into multiple disconnected arc segments. This issue, in addition to complex background, causes degradation in the perfor- mances of the existing ellipse detection methods in terms of either the detection accuracy or the execution time, often both. Moreover, efforts to improve detection accuracy often result in longer exe- cution time, while algorithmic effort s in reducing execution time often compromise the detection accuracy.

Five-dimensional Hough Transform (HT) is a classical method for detecting ellipses. Regardless of the robustness, the standard Hough transform [11] involves a large amount of computational cost. To improve the computational efficiency of HT, various HT- based approaches [12–16] , such as Randomized Hough Transform (RHT) [15] and Probabilistic Hough Transform (PHT) [16] ,have been developed. However, these variants of HT cannot reach video rate computation speeds due to the process of voting among nu- merous candidates.

Another class of ellipse detection approaches is edge segment detection techniques. Methods in this class exploit the connectivity between edge pixels to detect ellipses. The main steps of these methods are to extract arcs and then group them by exploiting ge- ometric or algebraic properties of ellipses. Specifically, edge arcs that may consist of an ellipse are found by variety of techniques such as the statistical regression method [19] , curve segmentation by fitting a set of short line segments on edges [20–26] , connec-tivity and curvature conditions [22,27] . Subsequently, these arcs are grouped according to the convexity-concavity [21,28,29] , arc curvature [21] , and geometric constraints [1,30–33] . Lastly, ellipses are fitted on the grouped arc candidates using RANSAC [19,32] or least squares fitting technique [28] . Prasad et al. [21] use arc con- vexity to determine search regions where suitable arc candidates for grouping may be located. Fornaciari et al. [28] use geometric constraints as a selection strategy of arcs belonging to the same ellipse and estimate parameters by decomposing the parameter space.

The edge segment detection methods are regarded as the most effective in detecting multiple ellipses in digital images and require less computational cost than HT-based methods. How- ever, because of dependence on the preciseness of arcs detected, the detection accuracy drops when these methods are applied to images containing complicated occluded contours.

This paper presents an ellipse detection method that combines the advantages of arc extraction and arc grouping. Arc detection involves important arc splitting steps, where our novel propo- sitions given in points 2 and 3 below help in precise arc detec- tion at low computation expense. Arc grouping using geometric constraints, discussed more in points 4 and 5 below, guarantees the effectiveness of ellipse detection and optimizes the compu- tation cost


%%% dong ntu 

%%% lu tip
ELLIPSE detection is a fundamental technique in image processing field and plays an indispensable role in shape
detection and geometric measurement.

%%% lu tip

\subsection{Ellipse Detection}
Geometric primitive detection is a fundamental problem in image processing. Among various geometric primitives, ellipse may be the most popular one that exists both in natural and man-made environments, and therefore plays an indispensable role in many applications like traffic signs recognition, 


In practice, the presence of noise substantially hinders the localization of the edge pixels on ellipses. Moreover, partial occlusions breaks ellipses' borders into discrete arc segments. These issues complicates the problem 
%and therefore causes degradation in the 
%causes degradation in the perfor- mances of the existing ellipse detection methods in terms of either the detection accuracy or the execution time, often both.
%compromise the detection accuracy


The earliest method relied on the generalized Hough Transform for detection ellipses. 
Regardless of the robustness, the standard Hough transform [11] involves a large amount of computational cost.
To improve the computational efficiency of HT, various HT- based approaches [12–16] , such as Randomized Hough Transform (RHT)


[11] R.O. Duda , P.E. Hart , Use of the Hough transformation to detect lines and curves in pictures, Commun. ACM 15 (1972) 11–15 .
[15] L. Xu , E. Oja , P. Kultanen , A new curve detection method: randomized Hough transform (RHT), Pattern Recognit. Lett. 11 (1990) 331–338 .
[8] R.A.McLaughlin,RandomizedHoughtransform:improvedellipsedetection
with comparison,PatternRecognitionLetters19(3–4)(1998)299–305

Another class of ellipse detection approaches is based on arc segments or edge chaining. 
Instead of working on each individual pixel, methods in this class work on arcs by linking pixels based on some connectivity principles.
exploit
The main steps of these methods are to extract arcs and then group them by exploiting ge- ometric or algebraic properties of ellipses.
Specifically, edge arcs that may consist of an ellipse are found by variety of techniques such as

Subsequently, these arcs are grouped according to the convexity-concavity [21,28,29] , arc curvature [21] , and geometric constraints [1,30–33] 

Lastly, ellipses are fitted on the grouped arc candidates using RANSAC [19,32] or least squares fitting technique [28]

Prasad et al. [21] use arc con- vexity to determine search regions where suitable arc candidates for grouping may be located. Fornaciari et al. [28] use geometric constraints as a selection strategy of arcs belonging to the same ellipse and estimate parameters by decomposing the parameter space.

effective

[21] D.K. Prasad , M.K. Leung , S.-Y. Cho , Edge curvature and convexity based ellipse detection method, Pattern Recognit. 45 (2012) 3204–3221 .
[28] M. Fornaciari , A. Prati , R. Cucchiara ,A fast and effective ellipse detector for embedded vision applications, Pattern Recognit. 47 (2014) 3693–3708
[30] H.I. Cakir , C. Topal , C. Akinlar , An occlusion-resistant ellipse detection method by joining coelliptic arcs, in: European Conference on Computer Vision, 2016, pp. 492–507 .


Many algebraic functions or geometric characteristics have been utilized for selection of arc segments to improve the efficiency of ellipse detectors.

Patraucean et al. [40] propose a line segment and elliptical arc detector without any parameter tuning based on a contrario theory
[40] V. P atr aucean , P. Gurdjos , R.G. von Gioi , Joint A Contrario ellipse and line de- tection, IEEE Trans. Pattern Anal. Mach. Intell. 39 (2017) 788–802 .

Jia et al. [42] use a projective invariant operator to significantly prune the undesired candidates and pick out elliptical ones. 
[42] Q. Jia , X. Fan , Z. Luo , L. Song , T. Qiu ,A fast ellipse detector using projective invariant pruning, IEEE Trans. Image Process. (2017) .

guarantees the effectiveness of ellipse detection

Based on the considerations above, we propose a new approach




1d rht
Since five parameters are required to fully define an
ellipse, a generalized Hough Transform (HT) needs a five
dimensional parameter space to detect an ellipse. That is very
memory and time consuming.

another 1d rht from ntu

Extracting elliptical objects from digital images is of fundamental
importance in shape recognition [1]. One of the best known methods
in extracting ellipses from images is the Hough Transform (HT).
The key concept behind the standard HT in extracting ellipses is to
define a mapping between the two dimensional image space and the
five dimensional parameter space. These five parameters are the coordinates
of the center point of the ellipse, the lengths of the major
and minor axes of the ellipse and the orientation of the major axis
with respect to the x-axis. Each data point of the image space is map
onto specific cells of the five dimensional accumulator, whereby the
associated parameters of the specific cells are chosen such that the
curve defined by these parameters passes through the data point. In
this aspect, the data points can be seen as voting for the parameters
of the ellipses found in the image. The votes in the cells are then accumulated.
After all data points of the image have been considered,
the local maxima of the accumulator correspond to the parameters
of the ellipses that are detected in the image.
The main advantage of the HT in extracting ellipse is its robustness
against discontinuous or missing data points. This is because
the HT does not require the connectivity of all the contour points
of an ellipse. Owing to this, the HT is well suited to detect ellipses
in the presence of moderate noise or in images having a cluttered
background. Unfortunately, the requirement for the five dimensional
accumulator places huge computational and storage constrains and
hence precludes practical applications. In light of this,

Prasad

Further,in real images,the
elliptic object may be occluded or hidden by other objects.
How-
ever,HT based ellipse detection methods have two main problems.
First problem is that HT is computation and memory intensive
because it uses a five-dimensional parameter space.For solving the
problem of five-dimensional parameters space,many methods were
proposed that split the five dimensional space into two or more
subspaces with lesser dimensionality and deal with each of the
subspaces in separate steps [13–17]. The most popular approach in
these methods was to find the centers of the ellipses using
geometrical theorems and Hough transform in the first step and
finding the remaining parameters of the ellipses in the second step.


eccv2016

Extracting ellipses from images is an important problem in computer vision and
has a diverse area of applications from object detection to pose estimation

Since the projections of circular objects appear as ellipse on the
camera image plane, ellipse detection is employed in many real life applications

There are many studies on ellipse detection found in the literature and
they are categorized in two groups, i.e. model-based and feature-based methods.
Although both approaches have pros and cons, many state-of-the-art algorithms
are feature-based methods and gives better results in terms of accuracy and
speed. Model-based methods fits a mathematical model to plain pixel information.
McLaughlin uses the famous Hough Transform (HT) for accurate ellipse
detection [17, mclaughlin]. A model-based search is a very slow operation for ellipse shape
since it needs to be performed in 5-dimensional parameter space.

Feature-based
approaches first extracts higher level geometric features, i.e. lines or arcs. In
recent years, preprocessing of arc and line segments with geometrical properties
are mostly investigated.

Prasad et al. propose a heuristic ellipse
detector based on convexity and edge curvature properties [21]. Fornaciari et al.
choose arcs strategically and compute parameters of ellipses with HT [6]. Arcs are
classified according to their convexity and then grouped to test if they compose
an ellipse or not

In this way, detection of
ellipses can be achieved even though arcs forming them are spread around
the image.

The very first step of the algorithm is extracting the edge segments from the
input image.

For n elliptical arcs extracted in the previous step, there are obviously 2n-1
potential combinations to join and detect ellipses.
Briefly, there is an exponential relationship between the detected elliptical
arcs and ellipse hypotheses. Therefore, it becomes a troublesome procedure to
test all arc combinations as the number of arcs grows

To reduce the number of arc combinations to test, we follow a proactive
way by analyzing all arcs in pairwise to determine whether any pair can be
part of the same ellipse or not. In other words, we try to find out whether any
two arcs are coelliptic or non-coelliptic. Once we determine any non-coelliptic
arc pair, we remove all arc subsets which includes that pair from entire 2n −
1 combinations to reduce the computation time.

Determining whether an arc
pair is non-coelliptic can be done in different ways.

To solve this problem, we use delimiting (start and
end) angles to validate whether

First, ellipse fit is applied to all arc pixels
in the current arc group and fitting error is computed. If the resulted error is
small, i.e. ≤ 2.0 pixels, we test the current arc group for the spanned angle ratio
(SAR) which is the ratio of the total length of arcs in the group over the perimeter
of the yielded ellipse.


Yaed

Ellipse detectionisthe
starting pointformanycomputervisionapplications,sinceellip-
tical shapesareverycommoninnatureandinhand-madeobjects.

wheels and road sign

The importanceofellipsedetectioninimageprocessingis
witnessed bythelargeamountofworkspresentintheliterature.

Most ofthemethodsforellipsedetectionrelyontheHough
Transform – HT (oritsvariants)toestimatetheparameters.Since
an ellipseisanalyticallydefined by five parameters,thesemethods
try toovercomethemainproblemofadirectapplicationof
standardHT,whichisa5Daccumulator.McLaughlin [10] relyon
the randomizedversionofHT(RHT)andaimatreducingthe
memory usageusingproperdatastructures.

Averycommonandmemoryefficient
approachhasbeenproposedbyXieandJi [12] and Chiaetal.
[13], wherefourparametersaregeometricallycomputed,estimat-
ing ina1Daccumulatorthelastone.

HT-basedmethods
greatlysufferfromnoise(whichincludesbothbackgroundnoise
and pointsbelongingtodifferentellipses)which “dirties” the
accumulator.Also,theyarecomputationallyintenseandrather
slow becauseofthevotingprocedureonahugenumberofedge
points combinations.

All aforementionedmethodsstarttheestimationfromsetsof
points, eventuallyselectedaccordingtosomekindofgeometric
constraints.However,whenconsideredunrelatedtoitsneighbors,
an edgepixeldoesnotcontributesignificantly toacorrectellipse
detection. Abettercharacterizationcouldbeachievedusingsetsof
connected edgepixels,i.e. arcs, whichcanbegeneratedbylinking
short straightlines [1,21–24], splittingtheedgecontour [25–28],
or validatingconnectededgepixels [29].

Theellipsesparameters
areobtainedusingellipse fitting methods [30,31] on areducedset
of arcs,whichareobtainedgroupingarcsaccordingtotheir
relativepositionandconstraintsonthecurvature [1,21,22,25,26],
or ellipse fitting error [23,24,27–29].

Toavoidthecombinatorialexplosion

Some arcs αk arenotsalientenoughtocharacterizeanellipse
and arereadilyremoved:veryshortarcs(NkoThlength) thatare
mainly duetonoiseandarcscontainingmostlycollinearpoints
(shortest sideofOBBkoThobb), thusnotbelongingtothecurved
boundary ofanellipse.

The selectionstrategy first selectspairswhosearcssatisfycon-
straints on(i)convexityand(ii)mutualposition


Thescore a in 0 1
summarizes howwellthepointsofthethreearcscomposing Ei fit
the boundaryoftheestimatedellipse:

bmvc integral chain

Prasad et al. [18] propose an edge curvature and convexity
based ellipse detection method which first extracts the canny edge map of an image
and chains the edge points into edge segments, and then fits line segments on each edge
segment to approximately represent the edge segments. After that, they use the convexity of
line segments to get elliptic arcs and further combine elliptic arcs into ellipses. Patraucean
et al. [17] propose an ellipse detection algorithm named ELSD. They first exploit a seed
growing scheme to find line-support regions, and further chain the line-support regions into
curve regions based on a convexity rule and a smoothness rule. After that, they estimate an
ellipse for each curve region with a fitting technique which merges the algebraic distance
with the gradient orientation. And finally, false detections are further eliminated by a contrario
validation technique. More recently, Fornaciari et al. [5] propose a very fast method
to detect ellipses which first splits the canny edge map into many short elliptic arcs, and then
classify these elliptic arcs into four types by their edge direction and convexity. They define
a candidate ellipse as a triplet, i.e. a set of three arcs that satisfy a set of criteria, then use
HT to determine the parameters of the ellipses. Finally, they validate the candidate ellipses
using a criterion based on the algebraic fitting error, and perform a clustering procedure to
deal with multiple detections.


Figure 4: Sample visual results


elsd
The critical parameters involved are the detection thresholds and the quantisation precision
of the accumulator array.

too permissive detection thresholds
or too coarse quantisation step introduce false positives, whilst over-restraining
detection thresholds or too ne quantisation step cause false negatives

aamed
The boundary of an ellipse are often splitted
into multiple arcs because of noise, overlap, and occlusion.
The combination of as many arcs as possible from the same
ellipse is an effective way to improve the detection accuracy

HT-based ellipse detection is the most popular method.
HT-based methods need to overcome the problems of huge
execution time and high memory usage.
Chia et al. [16]
used a one-dimensional (1D) accumulator and reduced the
memory storage and computational complexity by keeping
the advantages of the HT. Besides, they could achieve a
shorter execution time by using parallel computation.

In recent years, the methods based on edge-link have greatly
improved the detection performance. The main problem of
these methods is how to determine the elliptical arcs that
belong to the same ellipse.

Prasad et al. [25] proposed a method
based on edge curvature and convexity. The search region was
provided to find eligible contours for grouping the current
contour. Pˇatrˇaucean et al. [26] proposed a parameterless line
segment and elliptical arc detector with enhanced ellipse fitting
(ELSD). The number of false alarm (NFA) method in ELSD,
inspired by Von Gioi et al. [27], was used to control the false
elliptical arcs. Fornaciar et al. [9] provided a novel algorithm
called yet another ellipse detector (YAED). YAED segmented
edge contours into four quadrants and selected three arcs
from different quadrants to construct a candidate combination
with an innovative arc selection strategy. Wang et al. [10]
proposed a top-down fitting strategy to combine edges into
ellipses. The fitting process was sped up by an integral
chain. Jia et al. [12] proposed an ellipse
detection method based on the characteristic number (CNED),
which used a modified projective invariant to get rid of the
useless candidates. It performed well in execution time and
detection accuracy. Lu et al. [31] proposed an accurate and efficient ellipse detector
by arc-support line segments. The arc-support groups were
formed by iteratively linking the arc-support line segments that
may belong to the same ellipse. 

Given an arc , all the arcs that are adjacent to A(i) must be found. KD-Tree is used to find
It will waste much
time by traversing all arcs to find all adjacent arcs.

elsd thesis
A central problem in computer vision is the extraction (detection) of predefined geometric primitives
from geometric data. A geometric primitive is a curve or surface that can be described
by an equation with a number of free parameters; here, we are interested especially in ellipse
detection. Geometric data are an unordered list of points in two- or three-dimensional Cartesian
space. Such data are obtained through a variety of methods, either directly from images or by
using an edge detector in the 2D case, or from stereovision in the 3D case. The detection of
geometric primitives1 could be a prerequisite to solving other higher-level problems, like pose
determination, object recognition, or 3D scene reconstruction.

Existing geometric primitive detection algorithms can be roughly classified into two categories:
Hough-based and edge chaining methods. Most of them operate on edge maps, obtained
using an edge detector [Canny 1986].

Hough-Based Ellipse Detection Methods
Probably the most popular method for primitive detection is the generalisation of the Hough transform (HT)

The
basic idea of GHT is to quantise the transformation space into D-dimensional cells, obtaining
the so-called (grid) accumulators. Each transformed image point Ti is quantised, and then
votes for one of these cells. Cells whose number of votes yields a local maximum ( spike) in the
accumulator space and that is superior to some threshold, will be considered as valid detections.
A major advantage of GHT is its relative insensitivity to imperfect data, that is, noise
and occlusions can be handled, up to a certain level. However, in practice, this approach is
computationally inefficient, as the accumulators may become very large, and the execution
time is important.

randomised HT (RHT) [Xu 1990, McLaughlin 1998],
which is different from GHT in that it combines voting with geometric properties of analytical
curves. RHT takes advantage of the fact that some analytical curves can be fully determined by
a certain number of points on the curve. For example, an ellipse (or a circle) can be determined
by three points, as detailed in section 3.6.

RHT uses an iterative stochastic process, with a
fixed number of iterations, and, at each iteration, the algorithm randomly chooses three edge
points, which will vote for the ellipse passing through the three points. The accumulator array
in this case is not anymore a D-dimensional (grid) accumulator, but rather a variable-length
list, where each entry contains the parameters of an already voted ellipse, and a score. With
a newly estimated ellipse, if a similar3 ellipse is already present in the accumulator array, the
corresponding entry is updated to the mean of the two ellipses and the score is incremented by
one; otherwise, a new entry is added to the accumulator array, with the parameters of the new
ellipse and score one.

In our case, under the hypothesis of a
quasi-affine homography H, we implicitly assume that H maps circles to ellipses


fitting

Informally, conic fitting is the process of constructing a curve that fits well (or best ), according
to a predefined criterion, to a series of data.
As there exist methods that
use a subpixel precision, or normalised pixel coordinates, we will use the generic term points
when referring to pixel or subpixel measurements.

Many methods have been proposed in the literature to address the conic fitting problem. Two
important issues need to be considered by the conic fitting techniques, namely accuracy and
complexity. Usually, accurate methods are iterative, therefore costly, while the computationally
efficient ones may lack precision to a certain extent.

In the deterministic category, the simplest methods for conic fitting rely on some
basic operations performed on the input data. For example, the moment-based method has
been widely used as it provides a very simple, yet efficient technique for locating circular features
when images are taken under some favorable conditions.
This method
is sensitive to illumination variations, and above all, to occlusions.

With the least-squares (LS) formulation, one defines an error distance and tries to minimise
in a least-squares sense this distance between the input points and the fitted curve. We identify
two main classes of LS fitting techniques, namely algebraic and geometric fitting.

They are
differentiated through the definition of the error measure: the former account for methods that
seek to minimise an error distance which has a geometric meaning (e.g. Euclidean distance)
[Ahn 2001], whereas the latter use an algebraic expression
as error distance [Fitzgibbon 1999].

In the last decades, tremendous research work was concentrated on the LS formulation (3.5-3.6)
of the circle and ellipse fitting in images, and the problem seems to be settled from a theoretical
point of view.

Geometric Fitting of Circles and Ellipses
a) Geometric fitting of circles
Geometric fitting algorithms for circles commonly use the Euclidean distance as error-of-fit
in the minimisation problem
Ahn et al. [Ahn 2001] describe non-parametric algorithms for circle and ellipse fitting,

Solving a non-linear least-squares problem is costly, as in general, no closed-form solution
exists, and therefore an iterative procedure needs to be used. In the case of ellipse fitting,
another difficulty arises, namely the computation of the error-of-fit. Finding the orthogonal
distance between a point and an ellipse requires solving a high degree polynomial.
Ahn et al. [Ahn 2001] solve simultaneously two quadratic equations through
an iterative procedure. 

Fitzgibbon et al. [Fitzgibbon 1999] were the first to propose a direct (non-iterative) ellipsespecific
fitting. They show that the ellipticity constraint 4ac − b2 > 0 can be conveniently
incorporated into the constraint h(θ) in the quadratic form
The drawback of Fitzgibbon et al.’
approach is its tendency to underestimate the eccentricity of the ellipse when the estimation is
done on incomplete data.



Existing commonly used methods for ellipse detection can
be briefly grouped into two categories: 1) Hough Transform;
2) Edge Following.

The basic idea of HT for ellipse
detection is voting arbitrary edge pixels into 5D parameter
space. The local peak will occur when the corresponding bin
of accumulator exceeds a threshold of votes, which implies
for detecting an ellipse. But it is almost impractical to directly
apply HT in practice due to the heavy computation burden and
excessive consumption of memory. To alleviate these issues,
considerable improved methods are put forward.

Instead of transforming each edge point into
a 5D parameter space, Xu et al. [20] proposed a Randomized
Hough Transform (RHT) to detect curves, which randomly
chooses five edge pixels each time and maps them into a point
of the ellipse parameter space. McLaughlin [11] combined the
aforementioned two-stage decomposition method and RHT
at the aim of reducing the computation time and improving
the detection performance compared with the standard HT,
which becomes a baseline of ellipse detection method in the
literature (Fig. 1(c)).

c)). However, it is still not efficient enough in
practice and always generates false detections due to the lack
of novel validation strategies. Despite the simplicity of HT,
HT based ellipse detection methods suffer from the following
legacy problems: First, it is vulnerable in front of substantial
image noise and complicated real-world background due to
false peaks; Second, it requires much effort to tune the model
parameters, e.g. bin size and peak threshold.

The second well-known family of ellipse detection methods
is edge following, in which the connectivity between edge
pixels, convexity of arc segments and geometric constraints are
used. The general idea of edge following always starts from
computing the binary edge map and corresponding gradients
acquired by Canny or Sobel detector [21], [22] and then
refining the arc segments from the binary edge for the ellipse
fitting.

The
ellipse detector proposed by Prasad et al. [12] incorporates the
edge curvature and convexity to extract smooth edge contours
and performs a 2D HT to rank the edge contours in a group
by the relationship scores for the better generation of ellipse
hypotheses. But it also suffers from a long computation time,
as shown in Fig. 1(d).
Another stream of edge following methods tries to extract
arc segments from binary edge directly and prunes straight
edges for the purpose of fast detection speed. The ellipse
detector proposed by Fornaciari et al. [13] assigns a bounding
box for each arc, removes the straight edges and determines
the convexity of the arc by comparing the areas of region
under and over the arc. In addition, this method accelerates
the detection process by utilizing the property of that ellipse
center should be colinear to the midpoints of parallel chords.
However, it raises the detection speed at the cost of localization
accuracy and robustness (Fig. 1(e)). Recently, the method
presented by Jia et al. [14] inherits [13] and uses the similar
convexity classification approach, but the difference lies in
that [14] filters straight edges efficiently by calculating the
edge connected component’s characteristic number, which is a
kind of projective invariant being able to distinguish the lines
and conic curves within images. [14] is fast and yet prone
to generate duplicates due to the absence of novel clustering
(Fig. 1(f)). In addition, both [13] and [14] require at least three
arc segments to recover the ellipse model, which might disable
the algorithms when handling the incomplete ellipses.

Some researchers generalize the LS detection method to be
a multi-functional detector which can jointly detect the LS
and elliptic arcs. ELSDc proposed by P˘atr˘aucean et al. [15]
uses an improved LSD [26] version for detecting LS, and
then iteratively searches the remaining LSs from the start and
end points of the detected LS. Eventually, both LS detection
and grouping tasks are established simultaneously. Notably,
ELSDc stands out other methods by detecting LSs from the
greyscale image instead of binary edge such that abundant
gradient and geometric cues can be fully exploited.

\bibliography{gdop} 
\bibliographystyle{ieeetr}

\end{document}