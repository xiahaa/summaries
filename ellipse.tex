\documentclass[a4paper]{report}
\usepackage{lipsum}
\usepackage{tikzpagenodes}
\usepackage{pgfplots}
\usepackage{tikz}
\usepackage{tikz-3dplot}
\usetikzlibrary{arrows,decorations.pathmorphing,backgrounds,positioning,fit,matrix}
\pgfplotsset{compat=1.8}
\usepackage{graphics} % for pdf, bitmapped graphics files
\usepackage{epsfig} % for postscript graphics files
\usepackage[colorlinks=true,citecolor=green]{hyperref}
\usepackage{cite}
\usepackage{amsmath,amssymb,amsfonts}
\usepackage{algorithmic}
\usepackage{graphicx}
\usepackage{url}
\usepackage{cite}
\usepackage{bm}
\usepackage{pbox}
\usepackage{siunitx,booktabs,etoolbox}

\def\BibTeX{{\rm B\kern-.05em{\sc i\kern-.025em b}\kern-.08em
    T\kern-.1667em\lower.7ex\hbox{E}\kern-.125emX}}

\begin{document}

\section{Introduction}
Geometric primitive detection is a fundamental problem in image processing. Among various geometric primitives, ellipse may be the most popular one that exists both in natural and man-made environments, and therefore plays an indispensable role in many applications like traffic signs recognition, 3D scene reconstruction
In practice, the presence of noise substantially hinders the localization of the edge pixels on ellipses. Further, partial occlusions splits ellipses' borders into discrete arc segments. These issues complicates the problem, which makes detecting ellipses efficiently and accurately a challenging task in practice. 
%and therefore causes degradation in the 
%causes degradation in the perfor- mances of the existing ellipse detection methods in terms of either the detection accuracy or the execution time, often both.
%compromise the detection accuracy
\subsection{Ellipse Detection}
There are many studies on ellipse detection found in the literature and they can be classified in two categories, i.e. Hough Transform based and edge linking methods.

Although both approaches have pros and cons, many state-of-the-art algorithms
are feature-based methods and gives better results in terms of accuracy and
speed.

In recent years, preprocessing of arc and line segments with geometrical properties are mostly investigated.

In recent years, the methods based on edge-link have greatly
improved the detection performance. The main problem of
these methods is how to determine the elliptical arcs that
belong to the same ellipse.

\subsubsection{Hough Transform based Methods}
The earliest method relied on the generalized Hough Transform for detection ellipses. Briefly,
the key concept of the HT is to cast the detection problem in the image space as a voting problem in the parameter space. More specifically, HT first establishes an accumulator in the parameter space. In the case of ellipse detection, HT needs all pixels to vote in a five dimensional parameter accumulator, i.e. the center, the major axis, the minor axis, and the orientation of the ellipse. Then, each cell of the accumulator will be voted by the pixels who lie on the primitive corresponding to the cell. 
After voting, the cells whose votes are local maxima and over a threshold correspond to the detected ellipses' parameters.
%with a certain number votes 
%\todo{a figure}

Despite its robustness to discontinuous data points, the HT based method is very time and memory consuming. 
%In light of this,
To improve the computational efficiency, some variants of HT were proposed with Randomized Hough Transform (RHT) being the most well known. RHT based methods randomly sample a certain number (3 for ellipse) of points to determine the parameters of the primitive and then vote the computed parameters with other points. 
[Xu 1990, McLaughlin 1998].
RHT alleviates the memory demands by replacing the grid accumulator with a variable-length list which grows its length only when a new entry is inserted.
%They work by splitting the five dimensional space into two or more subspaces with lesser dimensionality and deal with each of the subspaces in separate steps
%Chia et al. [16] used a one-dimensional (1D) accumulator and reduced the memory storage and computational complexity by keeping the advantages of the HT. Besides, they could achieve a shorter execution time by using parallel computation.
Despite these improvements, they are still not efficient enough in practice. 
% Apart from efficiency,  
% the performance of all HT based methods rely on the tuning of the quantisation precision of the accumulator array.
% Second, it requires much effort to tune the model parameters, e.g. bin size and peak threshold.



% @article{xu1990new,
%  title={A new curve detection method: randomized Hough transform (RHT)},
%  author={Xu, Lei and Oja, Erkki and Kultanen, Pekka},
%  journal={Pattern recognition letters},
%  volume={11},
%  number={5},
%  pages={331--338},
%  year={1990},
%  publisher={Elsevier}
%}
%@article{mclaughlin1998randomized,
%  title={Randomized Hough transform: improved ellipse detection with comparison},
%  author={McLaughlin, Robert A},
%  journal={Pattern Recognition Letters},
%  volume={19},
%  number={3-4},
%  pages={299--305},
%  year={1998},
%  publisher={Elsevier}
%}
%@article{duda1972use,
%  title={Use of the Hough transformation to detect lines and curves in pictures},
%  author={Duda, Richard O and Hart, Peter E},
%  journal={Communications of the ACM},
%  volume={15},
%  number={1},
%  pages={11--15},
%  year={1972},
%  publisher={ACM New York, NY, USA}
%}
%@article{ballard1981generalizing,
%  title={Generalizing the Hough transform to detect arbitrary shapes},
%  author={Ballard, Dana H},
%  journal={Pattern recognition},
%  volume={13},
%  number={2},
%  pages={111--122},
%  year={1981},
%  publisher={Elsevier}
%}

\subsubsection{Edge linking methods}
%critical parameters
Another class of ellipse detection approaches is based on arc segments or edge chaining. 

Instead of working on each individual pixel, methods in this class work on arcs by linking pixels or line segments based on some connectivity principles. 

Next, the extracted arcs are grouped by exploiting convexity and positional constraints.
%Subsequently, these arcs are grouped according to the 

Specifically, edge arcs that may consist of an ellipse are found by variety of techniques such as
convexity-concavity [21,28,29] , arc curvature [21] , and geometric constraints [1,30–33] 
%[21] D.K. Prasad , M.K. Leung , S.-Y. Cho , Edge curvature and convexity based ellipse detection method, Pattern Recognit. 45 (2012) 3204–3221 .
%[28] M. Fornaciari , A. Prati , R. Cucchiara ,A fast and effective ellipse detector for embedded vision applications, Pattern Recognit. 47 (2014) 3693–3708
%[30] H.I. Cakir , C. Topal , C. Akinlar , An occlusion-resistant ellipse detection method by joining coelliptic arcs, in: European Conference on Computer Vision, 2016, pp. 492–507 .







Lastly, ellipses are fitted on the grouped arc candidates using RANSAC [19,32] or least squares fitting technique [28]

The ellipses parameters are obtained using ellipse fitting methods [30,31] on a reduced set of arcs, which are obtained grouping arcs according to their relative position and constraints on the curvature [1,21,22,25,26], or ellipse fitting error [23,24,27–29].


exploit


effective

%[40] V. P atr aucean , P. Gurdjos , R.G. von Gioi , Joint A Contrario ellipse and line de- tection, IEEE Trans. Pattern Anal. Mach. Intell. 39 (2017) 788–802 .
%
%[42] Q. Jia , X. Fan , Z. Luo , L. Song , T. Qiu ,A fast ellipse detector using projective invariant pruning, IEEE Trans. Image Process. (2017) .

guarantees the effectiveness of ellipse detection

Based on the considerations above, we propose a new approach



Prasad et al. [18] proposed to extract arcs by chaining line segments within a tolerate curvature change and then group detected arcs using convexity constraint. 
But it also suffers from a long computation time,
  

%(YAED)
More recently, Fornaciari et al. [5] propose a very fast method which first extracts arc contours and split them into four quadrants based on their edge direction and convexity. Then, they establish valid triplets (i.e. three different arcs) by grouping arcs from proper quadrants and checking some positional constraints. HT is used to estimate the ellipse parameters for all valid triplets. Finally, the candidate ellipses are validated based on the algebraic fitting error. To suppress duplicate detections, they cluster similar ellipses by only keeping the one with the minimum fitting error. This method was later improved by Jia et al. [12] where they proposed two projective invariants that can be used to prune invalid groups of arcs before estimating parameters using HT.%CNED
In addition, both [13] and [14] require at least three arc segments to recover the ellipse model, which might disable the algorithms when handling the incomplete ellipses.


Many algebraic functions or geometric characteristics have been utilized for selection of arc segments to improve the efficiency of ellipse detectors.

%By extending LSD [26], 
Patraucean et al. [26] proposed an ellipse detection algorithm named ELSD. After extracting line segments, they chain neighbouring line segments to form arcs based on a smoothness rule. After that, ellipses are fitted and validated by a-contrario technique.  
% a convexity rule and a smoothness rule


Lu et al. [31] proposed an accurate and efficient ellipse detector by arc-support line segments. The arc-support groups were formed by iteratively linking the arc-support line segments that may belong to the same ellipse. 


% Another stream of edge following methods tries



\section{Method}
briefly

In this way, detection of ellipses can be achieved even though arcs forming them are spread around the image.

The very first step of the algorithm is extracting the edge segments from the input image.

For n elliptical arcs extracted in the previous step, there are obviously 2n-1 potential combinations to join and detect ellipses.
Therefore, it becomes a troublesome procedure to test all arc combinations as the number of arcs grows

To reduce the number of arc combinations to test, we follow a proactive way by analyzing all arcs in pairwise to determine whether any pair can be part of the same ellipse or not. In other words, we try to find out whether any
two arcs are coelliptic or non-coelliptic. Once we determine any non-coelliptic arc pair, we remove all arc subsets which includes that pair from entire 2n-1 combinations to reduce the computation time.

Determining whether an arc pair is non-coelliptic can be done in different ways.

To avoid the combinatorial explosion


First, ellipse fit is applied to all arc pixels in the current arc group and fitting error is computed. 
If the resulted error is small, i.e. <= 2.0 pixels, we test the current arc group for the spanned angle ratio
(SAR) which is the ratio of the total length of arcs in the group over the perimeter of the yielded ellipse.


Some arcs are not salient enough to characterize an ellipse
and are readily removed: very short arcs that are
mainly due to noise and arcs containing mostly collinear points), thus not belonging to the curved
boundary of an ellipse.


The score a in 0 1
summarizes how well the points of the three arcs composing Ei fit the boundary of the estimated ellipse:


Yaed

The selection strategy first selects pairs whose arcs satisfy constraints on(i)convexity and(ii)mutual position

Figure 4: Sample visual results


Wang et al. [10] proposed a top-down fitting strategy to combine edges into
ellipses. The fitting process was sped up by an integral
chain.

aamed



Given an arc , all the arcs that are adjacent to A(i) must be found. KD-Tree is used to find
It will waste much
time by traversing all arcs to find all adjacent arcs.

elsd thesis


In our case, under the hypothesis of a
quasi-affine homography H, we implicitly assume that H maps circles to ellipses




\subsection{fitting}
Ellipse fitting estimates the ellipse parameters by fitting an optimum (with respect to a criterion) ellipse to the input data.
Most ellipse fitting methods in the literature use 2 Norm as the criterion, which forms a least-squares problem.
%Many methods have been proposed in the literature to address the ellipse fitting problem. 
Based on how they solve this problem, they can be classified as analytical or iterative. Generally, iterative methods are more accurate, but less efficient than analytical solutions.
Depending on how the error is defined, they can be categorized into two classes: methods minimizing the algebraic distance and methods minimizing the geometric distance. Fitting with geometric distance is generally believed to be more accurate than fitting with algebraic distance.
%under some favorable conditions
[Fitzgibbon 1999]
[Ahn 2001]
%minimise an error distance which has a geometric meaning
% whereas the latter use an algebraic expression
Ahn et al. [Ahn 2001] describe non-parametric algorithms for circle and ellipse fitting.


Solving a non-linear least-squares problem is costly, as in general, no closed-form solution
exists, and therefore an iterative procedure needs to be used. In the case of ellipse fitting,
another difficulty arises, namely the computation of the error-of-fit. Finding the orthogonal
distance between a point and an ellipse requires solving a high degree polynomial.
Ahn et al. [Ahn 2001] solve simultaneously two quadratic equations through
an iterative procedure. 

Fitzgibbon et al. [Fitzgibbon 1999] were the first to propose a direct (non-iterative) ellipsespecific
fitting. They show that the ellipticity constraint 4ac-b2>0 can be conveniently
incorporated into the constraint in the quadratic form
The drawback of Fitzgibbon et al.’
approach is its tendency to underestimate the eccentricity of the ellipse when the estimation is
done on incomplete data.






\bibliography{gdop} 
\bibliographystyle{ieeetr}

\end{document}