\documentclass[a4paper]{report}
\usepackage{lipsum}
\usepackage{tikzpagenodes}
\usepackage{pgfplots}
\usepackage{tikz}
\usepackage{tikz-3dplot}
\usetikzlibrary{arrows,decorations.pathmorphing,backgrounds,positioning,fit,matrix}
\pgfplotsset{compat=1.8}
\usepackage{graphics} % for pdf, bitmapped graphics files
\usepackage{epsfig} % for postscript graphics files
\usepackage[colorlinks=true,citecolor=green]{hyperref}
\usepackage{cite}
\usepackage{amsmath,amssymb,amsfonts}
\usepackage{algorithmic}
\usepackage{graphicx}
\usepackage{url}
\usepackage{cite}
\usepackage{bm}
\usepackage{pbox}
\usepackage{siunitx,booktabs,etoolbox}

\def\BibTeX{{\rm B\kern-.05em{\sc i\kern-.025em b}\kern-.08em
    T\kern-.1667em\lower.7ex\hbox{E}\kern-.125emX}}

\begin{document}




\section{Method}

guarantees the effectiveness of ellipse detection

Based on the considerations above, we propose a new approach

exploit


effective

The main problem of
these methods is how to determine the elliptical arcs that
belong to the same ellipse.

briefly

In this way, detection of ellipses can be achieved even though arcs forming them are spread around the image.

The very first step of the algorithm is extracting the edge segments from the input image.

For n elliptical arcs extracted in the previous step, there are obviously 2n-1 potential combinations to join and detect ellipses.
Therefore, it becomes a troublesome procedure to test all arc combinations as the number of arcs grows

To reduce the number of arc combinations to test, we follow a proactive way by analyzing all arcs in pairwise to determine whether any pair can be part of the same ellipse or not. In other words, we try to find out whether any
two arcs are coelliptic or non-coelliptic. Once we determine any non-coelliptic arc pair, we remove all arc subsets which includes that pair from entire 2n-1 combinations to reduce the computation time.

Determining whether an arc pair is non-coelliptic can be done in different ways.

To avoid the combinatorial explosion


First, ellipse fit is applied to all arc pixels in the current arc group and fitting error is computed. 
If the resulted error is small, i.e. <= 2.0 pixels, we test the current arc group for the spanned angle ratio
(SAR) which is the ratio of the total length of arcs in the group over the perimeter of the yielded ellipse.


Some arcs are not salient enough to characterize an ellipse
and are readily removed: very short arcs that are
mainly due to noise and arcs containing mostly collinear points), thus not belonging to the curved
boundary of an ellipse.


The score a in 0 1
summarizes how well the points of the three arcs composing Ei fit the boundary of the estimated ellipse:


Yaed

The selection strategy first selects pairs whose arcs satisfy constraints on(i)convexity and(ii)mutual position

Figure 4: Sample visual results


Wang et al. [10] proposed a top-down fitting strategy to combine edges into
ellipses. The fitting process was sped up by an integral
chain.

aamed



Given an arc , all the arcs that are adjacent to A(i) must be found. KD-Tree is used to find
It will waste much
time by traversing all arcs to find all adjacent arcs.

elsd thesis


In our case, under the hypothesis of a
quasi-affine homography H, we implicitly assume that H maps circles to ellipses




\subsection{fitting}
Ellipse fitting estimates the ellipse parameters by fitting an optimum (with respect to a criterion) ellipse to the input data.
Most ellipse fitting methods in the literature use 2 Norm as the criterion, which forms a least-squares problem.
%Many methods have been proposed in the literature to address the ellipse fitting problem. 
Based on how they solve this problem, they can be classified as analytical or iterative. Generally, iterative methods are more accurate, but less efficient than analytical solutions.
Depending on how the error is defined, they can be categorized into two classes: methods minimizing the algebraic distance and methods minimizing the geometric distance. Fitting with geometric distance is generally believed to be more accurate than fitting with algebraic distance.
%under some favorable conditions
%[Fitzgibbon 1999]
%[Ahn 2001]
%minimise an error distance which has a geometric meaning
% whereas the latter use an algebraic expression


The most representative approach for direct ellipse fitting was proposed by Fitzgibbon et al. [13]. The authors formulated the ellipse fitting problem as a least square problem with an ellipticity constrain, which can be solved using generalized eigenvalue decomposition. This method is very efficient, but tends to underestimate the eccentricity of the ellipse when the estimation is
done on incomplete data.


To tackle the weaknesses of algebraic fitting, fitting by minimizing geometric distances was developed. Unfortunately, determine the geometric distance of a point to an ellipse is not easy. There is not a closed-form solution for the orthogonal distance from one point to the ellipse. Therefore, the geometric fitting has to be done iteratively. 
The most well known method was proposed by Ahn et al. [Ahn 2001].
Ahn’s method uses two nested
iterative loops.The outer loop considers the data of pixels as a
whole and uses gradient descent approach to optimize the esti-
mated geometric parameters(namely,coordinates of the center,
angle made by the major axis with x- axis, and the lengths of
semimajor and semiminor axes).The inner loop is executed for each
pixel and the estimation of a point on the ellipse that is nearest to
the considered pixel is optimized iteratively.Ahn’s method uses
geometric parameters as the driving factors of the algorithmin both
the loops.

While being effective and
robust, Anh’smethod is computationally expensive owing to the iterative optimization for each pixel with in the outer loop.
%[1] Ahn, S.J., Rauh, W., Warnecke, H.J.: Least-squares orthogonal distances fitting of
%circle, sphere, ellipse, hyperbola, and parabola. Pattern Recogn. 34(12), 2283–2303
%(2001)
%
%[2] P.L.Rosin,Anoteontheleastsquaresfittingofellipses,PatternRecognition
%Letters 14(1993)799–808.


On these grounds, in this thesis, we use the direct method~\cite{fitzgibbon1999direct}  for fast fitting in almost all cases and only switch to iterative method  


\bibliography{gdop} 
\bibliographystyle{ieeetr}

\end{document}