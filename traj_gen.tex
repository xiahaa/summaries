\documentclass[a4paper]{report}
\usepackage{lipsum}
\usepackage{tikzpagenodes}
\usepackage{pgfplots}
\usepackage{tikz}
\usepackage{tikz-3dplot}
\usetikzlibrary{arrows,decorations.pathmorphing,backgrounds,positioning,fit,matrix}
\pgfplotsset{compat=1.8}
\usepackage{graphics} % for pdf, bitmapped graphics files
\usepackage{epsfig} % for postscript graphics files
\usepackage[colorlinks=true,citecolor=green]{hyperref}
\usepackage{cite}
\usepackage{amsmath,amssymb,amsfonts}
\usepackage{algorithmic}
\usepackage{graphicx}
\usepackage{url}
\usepackage{cite}
\usepackage{bm}
\def\BibTeX{{\rm B\kern-.05em{\sc i\kern-.025em b}\kern-.08em
    T\kern-.1667em\lower.7ex\hbox{E}\kern-.125emX}}

\begin{document}
Planning dynamically feasible and collision free trajectories in dense indoor environments is investigated in [Aadam Bry]. However, it requires the environment map is known.

time interval of the trajectory 

To demonstrate and validate the performance of the prosed method, simulations in various scenarios are conducted.
The virtual environment are built, see Figures.

Quadrotor UAVs are relatively simple, affordable and easyto-
manipulate aerial robotic systems. Even if they have turned
out to be reliable and efficient testbeds in both civilian and
commercial missions in the past two decades, e.g., wildlife
conservation [1], midair collision surveillance [2] and unknown
environment exploration [3], the space for mounting different
onboard sensors, payload capacity, flight time and area of a
single quadrotor is still low.

In this paper, the formation landing problem of quadrotor
UAVs has been discussed, path planning for avoiding obstacles
has also been involved. In literature, many algorithms can be

Basically, the only interaction with external environments
in the RRT is the collision detection of vehicle
navigaiton, which is transferable to different situations. The
RRT is usually considered as a versatile algorithm which can
deal with varaint configurations.

Test scenario plotted in

Research on path planning can be used to tackle with this
concern to some extent. Global path planning algorithms such as
A*, Dijkstra or Probabilistic Roadmaps (RPM) can be used to
generate a path between the starting point and the destination
that avoids obstacles [14], [15]. However, vehicle model and its
dynamics were not considered in these algorithms, so that there
is no guarantee that these path or trajectory can be well tracked

In this paper, we consider the unicycle-type mobile robot and
propose a new trajectory generation algorithm. This algorithm
takes vehicle model and speed limitations into consideration and
guarantees the tractability of the generated reference trajectory.
In addition, we also implement a control law specially designed
for these constraints, so that an optimal trajectory tracking
performance can be achieved. 

To generate a path or a trajectory within an environment with
constraints such as walls and obstacles, global path planning
algorithm such as A* and D* algorithm are usually used. In A*
algorithm, the map is modeled as a lattice of nodes 𝑛, and three
types of cost functions $f(n)$, $g(n)$ and $h(n)$ are defined. By
searching around the nodes and calculating these three cost
functions iteratively, an optimal path can be determined between
the starting and the ending points. The disadvantage of A*
algorithm is that it works only for static maps but cannot deal
with dynamic change in maps or obstacles. In D* (or Dynamic
A Star) algorithm, further improvements are made in this regard.
When the robot starts moving, the adjacent nodes along the
optimal path are continuously observed for possible changes
from the prior information. When changes (indicating moving
obstacles or changes in map info) are observed, using similar
concept as in A*, the path estimation cost function $h(n)$ is
adjusted by a weights function $c(X,Y)$, an A*-like process is
evaluated again for the adjacent nodes, and a new path can be
generated.
To further decrease the computational cost, sample-based
planning methods such as Rapidly-exploring Random Trees
(RRT) and Probabilistic Roadmaps (RPM) were proposed in
[16], [17]. These methods can be viewed as growing a tree from
a robot's starting point until one of its branches hits the goal. The
tree branches are created by randomly sampling probabilistic
points. Tests are run for collision-free check and an optimal path
can be found by finding the shortest path.

d the generated path may contain sharp turns,
zig-zag path

. Addressing these problems, we present an
approach for planning safe, dynamically feasible and time
efficient flight paths using cubic splines. Trying to reduce
trajectory time, we consider the influence of path smoothing
and refinement measures on the quality of the resulting flight
paths and on the reliability of the planning procedure.

For online navigation through a priori unknown environment,
unmanned rotorcraft must be able to react instantaneously
to the detection of obstacles. While reactive obstacle
avoidance algorithms by themselves are inherently suboptimal
and incomplete, it is hard to meet runtime constraints
with more complex motion planning algorithms. Decoupling
the planning of collision free and dynamically feasible trajectories
into sequential planning steps is a practical solution
to make the complex planning problem efficiently solvable
in order to meet runtime constraints. 

Setting the focus on these problems, we present a highly
decoupled flight path planning approach for unmanned rotorcraft
navigating in a priori unknown environment. We combine
sampling-based path planning, cubic spline interpolation
and path refinement techniques in order to plan safe and
time efficient flight paths, that account for the rotorcraft’s
kinematic constraints. While maintaining the efficiency and
flexibility of this decoupled planning approach, we overcome
its deficiencies through carefully separating each processing
step and its effect on the flight path geometry. The presented
approach utilizes the computational efficiency of samplingbased
path planning to avoid obstacles and cubic spline
interpolation to generate smooth paths. Iterative path refinement
is performed in order to maintain the required obstacle
clearance of the smoothed path and to account for the current

In
[10] an algorithm generating minimum snap trajectories for
quad-rotors is presented allowing constraints on position,
velocity, acceleration and control inputs while minimizing a
cost functional. 

If the smoothed path deviates too much from the
linear path segments, the obstacle clearance may be violated
compromising the global planner’s property of completeness.
This may be solved through iteratively performing collision
checks and locally restricting the smoothed path’s deviation
by inserting support points or adjusting parameter used
during interpolation as described in

. As the focus of this work is
set on minimizing trajectory times, the euclidean distance
is used as a cost function approximating the traversal time
without having to consider system dynamics.

In [11] parametric cubic splines were proposed for the
interpolation of linear paths in order to minimize trajectory
time. 

The
choice of parametrization has a great influence on the shape
of the resulting curve.

In order to track a nonlinear flight path with an upper
bound on the total body fixed acceleration, amax, the total
velocity along the flight path, Vk,max, has to be restricted
in order to be able to follow the curved path. 

The deviation dspl of the smoothed path from the
linear path segments may lead to a violation of the obstacle
clearance dclear, which was used by the samplingbased
planner

If a spline curve is found to
violate the deviation limit in an interval $[\tau_i
, \tau_{i+1}]$, support
points are inserted iteratively in this interval, following a
strategy described below, until the violation is resolved

Using these values as a starting point, we
iteratively decrease the start point curvature, until a feasible
velocity profile can be found.

RESULTS AND DISCUSSION
In this section simulation results of the online trajectory
time reduction approach are presented.

Experimental Setup

SIMULATION PARAMETERS USED FOR BENCHMARKING
Parameter Value
amax 0.5 m/s
2
rmax 180 °/s
Vk,xy,max 3 m/s
Vk,z,max 1.5 m/s

\end{document}