\documentclass[a4paper]{report}
\usepackage{lipsum}
\usepackage{tikzpagenodes}
\usepackage{pgfplots}
\usepackage{tikz}
\usepackage{tikz-3dplot}
\usetikzlibrary{arrows,decorations.pathmorphing,backgrounds,positioning,fit,matrix}
\pgfplotsset{compat=1.8}
\usepackage{graphics} % for pdf, bitmapped graphics files
\usepackage{epsfig} % for postscript graphics files
\usepackage[colorlinks=true,citecolor=green]{hyperref}
\usepackage{cite}
\usepackage{amsmath,amssymb,amsfonts}
\usepackage{algorithmic}
\usepackage{graphicx}
\usepackage{url}
\usepackage{cite}
\usepackage{bm}
\usepackage{pbox}
\usepackage{siunitx,booktabs,etoolbox}

\def\BibTeX{{\rm B\kern-.05em{\sc i\kern-.025em b}\kern-.08em
    T\kern-.1667em\lower.7ex\hbox{E}\kern-.125emX}}

\begin{document}

\section{Introduction}

Traditional fiducial detectors work well in idea conditions. 

However, they are not flexible in practice. Change of the fiducial type or size often requires tedious re-adjustments of some related thresholds which make the algorithms work properly.



Although the modern state of the arts can solve the tracking and recognition tasks very well, most of them rely on the acceleration of using high-performance GPU. In case that the computational resource is limited, their efficiency will drop significantly.

Under such circumstances, this article proposes and develops a lightweight approach to realize a long-term, real-time, and automatic positioning, using only low-cost onboard sensors and processors.

Considering that the fiducials only occupy small portions of the whole image, it is not worthwhile to detect them by analysing the whole frame, while simply downsample the image and then feed it to a detector may deteriorate the accuracy due to the loss of information. Consequently, in this thesis, the fiducial detection is solved in two steps: first the ROIs are detected from a coarse image and then an arc based ellipse detector is used to detect valid ellipses. This scheme improves efficiency as well as maintains accuracy. 
%The reason for dividing the whole process in two steps is for 

For ROI detection, the following framework is applied. Initially, the template matching is used to extract a number of possible candidate ROIs. A cascaded classifier is used to verify the valid ROIs and reject others. When an valid ROI is found, an associated tracker is also initialized so that visual tracking can be applied in the new coming frames. 

Indeed, the extraction and classification of ROIs can be combined in one step by using the convolutional implementation of sliding windows. Nonetheless, for the sake of efficiency, classical template matching is used. The use of tracking rather than running frame based detection is due to the same reason. As shown in the experiments, this framework can realize a performance of 40-60 Hz, which is highly sufficient for processing the capture video steams(30 Hz).



\section{Related Work}

\subsection{Object Detection}

Besides classical appearance based methods, another category of classical methods rely on handcrafted features and machine learning techniques. Perhaps the most famous method belonging to this category is the algorithm developed by Dalal
and Triggs (2005) for pedestrian detection. It extracts histogram of oriented gradients (HOG) descriptors and feed them into a support vector machine (SVM) based classifier.

%cluster each of these datasets into six separate clusters using k-means and then fit PCA subspaces to each of the resulting 12 clusters

%has also received widespread attention


Ever since its success on imagenet competition, deep learning and convolutional neural networks (CNNs) have revolutionized the field of computer vision over the last
years. It is now almost the standard tool for a large variety of computer vision tasks, in particular for object detection, classification, and segmentation. 

Although existing network have varying architectures, there are some common elements that are used in most networks, e.g. the convolutional layer, the Rectified Linear Unit (ReLU) activation function, etc. A comprehensive summary of deep learning related knowledge and techniques can be found in [20].
[45] J. Redmon, S. Divvala, R. Girshick, and A. Farhadi. “You only look
once: Unified, real-time object detection”. In: Proceedings of the IEEE
conference on computer vision and pattern recognition. 2016, pp. 779–
788.
[20] I. Goodfellow, Y. Bengio, and A. Courville. Deep learning. MIT press,
2016.
[19] X. Glorot, A. Bordes, and Y. Bengio. “Deep sparse rectifier neural networks”.
In: Proceedings of the Fourteenth International Conference on
Artificial Intelligence and Statistics. 2011, pp. 315–323.



New network architectures, layer types, and methods to improve the training efficiency and the performance of the CNNs are proposed at a very rapid pace.

In earlier days, the use of deep learning networks is somehow constrained by the availability of sufficient amount of data and computational resources. However, with the help of transfer learning, this has now been solved to some extents. Pretrained models can be applied for new application tasks by fine-tuning only a few task-related layers, which not only saves large amount of training time but also avoids the over-fitting problem.


\subsection{CNN}






\subsection{On Fiducial}
Compared with detection of general objects, detection of fiducial marker is an relatively easy task. Classical methods make use of image processing methods to extract certain geometric shapes and then verified the correctness based on geometric attributes or coding information. 

As mentioned in the introduction, the general process for square marker detection comprise: edge detection or thresholding, contour extraction and filtering for finding quadrangles, verification of internal codes. 
%cite Automaticgenerationanddetectionofhighlyreliable fiducial markers under occlusion


it is very slow for our real-time purposes.
In this work,we have opted for a local adaptive thresholding approach 
Suzuki and Abe [36] algorithm. 
Douglas–Peucker [37] algorithm. 

35] J. Canny,Acomputationalapproachtoedgedetection,IEEETrans.Pattern
Anal. Mach.Intell.8(6)(1986)679–698.
[36] S. Suzuki,K.Abe,Topologicalstructuralanalysisofdigitizedbinaryimagesby
border following,Comput.Vis.Graph.ImageProcess.30(1)(1985)32–46.
[37] D.H. Douglas,T.K.Peucker,Algorithmsforthereductionofthenumberof
points requiredtorepresentadigitizedlineoritscaricature,Cartographica:
Int. J.Geogr.Inf.Geovis.10(2)(1973)112–122.
[38] N. Otsu,Athresholdselectionmethodfromgray-levelhistograms,IEEETrans.
Syst. ManCybern.9(1)(1979)62–66.



As discussed in introduction, detection using machine learning or deep learning techniques has now shown great advantage on noise and blur resilience~\cite{prasad2015motion}.


ChromaTag[19]modifies AprilTag to use colors for both can-didate detection and decoding (seeFig. 2g). The detection algorithmonly runs on areas where both violet and green are present, whichdecreases the detection time significantly.


\subsection{Tracking}
Tracking: Fiducial detection between successive video
frames can be considered as a tracking problem where the
tracked object is the fiducial. There is a very large body
of research dedicated to tracking and interested readers are
referred to [29] for a good survey

Visual tracking aims to track a target in subsequent frames, which has many practical applications. Generally speaking, tracking is simpler than detection because the information learned from the object will be reused in the consecutive frames. Moreover, tracking algorithm normally is much faster than detection algorithm since it only needs to precess a small patch around the object rather than the whole image.
% use fewer computational resources
Thanks to these merits, it is often preferred to perform tracking after a target is detected rather than running detection algorithm on every new frame. Visual tracking has attracted much attention in recent years, in particular after the releases of several public datasets. 
 
There are several factors, namely occlusion, appearance and scale variations, that make tracking a challenging task. A substantial number of works have been proposed to deal with those problems and track objects in complex motions.


In terms of the literatures,

Another representative work is the tracking-by-detection method [10], which attempts to solve the long-term object tracking problems.
The detector, tracker, and learner were integrated by TLD to robustly track a single object in cases of heavy occlusion, out-of-view, scale change, etc. 


[10] Z. Kalal, K. Mikolajczyk, and J. Matas, “Tracking-learning-detection,”
IEEE Trans. Pattern Anal. Mach. Intell., vol. 34, no. 7, pp. 1409–1422,
Dec. 2012.
[11] S. Hare, S. Golodetz, and A. Saffari, “Struck: Structured output tracking
with Kernels,” IEEE Trans. Pattern Anal. Mach. Intell., vol. 38, no. 10,
pp. 2096–2109, Dec. 2015.
[27] Y. Wu, J. Lim, and M. Yang, “Online object tracking: A benchmark,” in
Proc. IEEE Conf. Comput. Vision Pattern Recognit., 2013, pp. 2411–2418.


\section{Visual Preprocessing}:
This section explains the steps employed to automatically
detect the markers in an image(Fig. 5(a)).
\subsection{Extraction}
% here why deep learning based trackers are not in consideration
Our algorithm extracts multiscale salient regions

% note when compute the response, multiple a scale factor, see advanced image processing
We develop a saliency map pyramid with multiple layers, to enhance the robustness for
scale-varying objects. In this article, a two-layer architecture is adopted, where the input image for the second layer is resized to half the size of the first.

\subsection{Classification}

Transfer learning is not considered here mainly because of efficiency. xxx


%% classifier
Each dimension of the output vector denotes the probability of
belonging to the corresponding class. Generally, the proposal
will be treated as the corresponding class only if the maximal
probability is greater than a specified threshold.

A classical CNN is trained for classification. The encoder is composed of several cascaded convolutional layers which are activated by ReLU and then followed by max pooling. The classifier comprises a few fully connected layers

 The final stage of a classification network is a
softmax operation, which outputs the probabilities for each object class.


 


% Provided that the base task is general enough, the learned model only needs to be fine-tuned when training on the application task.  Second, there is a large gain in training time and thus saving computational resources when solving the application task.  on dedicated hardware.


\subsection{Tracking}
Similar to Detector, the real-time capability of the tracker is
the main performance to be considered. In this system, the the
kernel correlation filter (KCF) tracker [38] is employed owing
to its great real-time capability. As for multiobject tracking task,
multiple KCF trackers are assigned to objects, respectively.
[38] J. Henriques, R. Caseiro, and P. Martins, “High-speed tracking with
Kernelized correlation filters,” IEEE Trans. Pattern Anal. Mach. Intell.,
vol. 37, no. 3, pp. 583–596, Aug. 2014.


The strength of a correlation peak can be used to detect tracking failure, as proposed in~\cite{MOSSE}. However, empirically, it is found not very reliable to rely on that score. Therefore, a strong threshold is used as an alert of potential tracking failure. Strong, in this context, means a threshold higher than the nominal value, which serves as a sufficient but not necessary condition. In case of a potential tracking failure, the classification is conducted to verify the validity of tracked ROIs. Detection is relaunched if the verification cannot pass, which prevents the tracker drifting away by learning wrong background information.
%while the reclassification is conducted every 10 frames.

%The tracker follows the object within specified region-of-interest from frame to frame with lower computing consumption.


%% experiment, some quantatitive indicators are needed
2) Accuracy: The accuracy comparison was conducted
among oDTL, TLD, TM, and YOLO, and the results are presented
in Table II. The precision P and recall R are computed


about geometric pattern detection,

\section{Whycon}
One of the classical solution for detecting the circular fiducial is the Generalized Hough Transform (Ballard 1981) used for finding the parameters of a geometrical shape, which is unfortunately computationally demanding. 

Other classical solutions rely on image analysis.

For example, in [whycon], the authors proposed a detection algorithm that search the image for circular patterns using a combination of adaptive thresholding, region growing, and on-the-fly verification of geometric attributes. More specifically, the algorithm starts by finding a pixel that can be classified as black class based on an adaptive threshold. Once it is found, region growing algorithm is applied subsequently to extract a connected component from the image. Then, the component is verified for correctness based on some geometric attributes such as area and roundness (they have to lay within a threshold). If the black region passes the verification, a corresponding white region is grown from the center of the black region, which is again verified using attributes like concentricity and area ratio. After the white region passes these tests, the pattern is considered to be found and PCA is used to find the ellipse parameters.



%The verification of a component as the searched pattern is based on the statistical information, namely area, roundness, extracted from grown regions. 

The author further improves the efficiency by searching the pattern in the next frame from the center of the pattern detected in current frame. 

%In a typical situation, the algorithm processes only the area that is occupied by the pattern itself, which results in a significant performance boost.





within acceptable bounds


cite{JIRS}

In [16], a planar pattern consisting of a ring surround-
ing the letter \H" is used to obtain the relative 6DoF
robot pose with an on-board camera and IMU (Inertial
Measurement Unit) to resolve angular ambiguity. The
pattern is initially detected by binarization using adap-
tive thresholding and later processing for connected com-
ponent labeling. For classifying each component as be-
longing to the target or not, a neural network (multilayer
perceptron) is used. The input to the neural network is a
resized 14 x 14 pixel image. After testing for certain ge-
ometric properties, false matches are discarded. Positive
matches corresponding to the outer ring are processed
by applying the Canny edge detector and ellipse tting,
which allows computation of the 5DoF pose. Recognition
of the \H" letter allows to obtain the missing yaw angle.
The precision in 3D position is in the order of 1 cm to
7 cm depending on the target viewing angle and distance,
which was at the maximum around 1:5 m.

Finally, a widely used, simple and freely available cir-
cular target detector can be found in the OpenCV li-
brary. This "SimpleBlobDetector" class is based on tra-
ditional blob detection methods and includes several op-
tional post-detection ltering steps, based on character-
istics such as area, circularity, inertia ratios, convexity
and center color. While this implementation is originally
aimed for circular target detection, by tuning the pa-
rameters it is possible to nd elliptical shapes similar
to the ones proposed in the present work and thus it is
compared to the proposed implementation.

3 Pattern detection
The core of the proposed computationally efficient local-
ization system is based on pattern detection. Fast and
precise detection is achieved by exploiting properties of
the considered pattern that is a black and white roundel
consisting of two concentric annuli with a white central
disc, see Figure 1.
The low computational requirements are met by the
pattern detection procedure based on on-demand thresh-
olding and fl
ood fill techniques, and gathering statistical
information of the pattern on the 
y. The statistical data
are used in consecutive tests with increasing complexity,
which determine if a candidate area represents the de-
sired circular pattern.
The pattern detection starts by searching for a black
segment. Once such a segment is detected and passes the
initial tests, the segment detection for a white inner disc
is initiated at the expected pattern center.
Notice, that at the beginning, there is no prior infor-
mation about the pattern position in the image; hence,
the search for the black segment is started at a ran-
dom position. Later, in the subsequent detections, when
a prior pattern position is available, the algorithm starts
detection over this area. For a successfully re-detected
(tracked) pattern, the detection processes only pixels be-
longing to the pattern itself, which signicantly reduces
the computation burden. Since the method is robust (see
following sections for detection limits), tracking is gen-
erally successful and thus the method provides very high
computational performance.
After the roundel is detected, its image dimensions
and coordinates are identied. Then, its three-dimensional
position with respect to the camera is computed from its
known dimensions and camera re-projection techniques,
and its coordinates are transformed to a coordinate frame
dened by the user, see Section 4.

segmentation
First,
a black circular ring is searched for in the input im-
age starting at an initial pixel position p0. The adaptive
thresholding classies the processed pixel using an adap-
tively set value aa as either black or white. If a black pixel
is detected, the queue-based 
ood-ll algorithm proce-
dure is initiated to determine the black segment.
Once the 
ood ll is complete, the segment is tested
for a possible match of the outer (or inner) circle of
the pattern. At this point, these tests consist of a min-
imum size (in terms of the number of pixels belonging
to the segment) and a roundness measure within accept-
able bounds.
Hence, ini-
tial simple constraints can be validated quickly for a fast
rejection of false positives.
In the case where either test fails, the detection for
further segments continues by starting from the next
pixel position (i.e., a pixel at the position p0 + 1). How-
ever, no redundant computation is performed since the
previous segment is labeled with a unique identier.

Efficient thresholding
Since the segmentation looks only for black or white seg-
ments, the success rate of the roundel detection depends
on the threshold parameter $\tau$ , especially under various
lighting conditions. Therefore, we proposed to adaptively
update aa whenever the detection fails according to a bi-
nary search scheme over the range of possible values.
When the pattern is successfully detected, the thresh-
old is updated using the information obtained during
detection in order to iteratively improve the precision of segmentation:

The computationally intensive full image threshold-
ing is addressed by on-demand processing over each pixel
analyzed during the detection. At the very rst access,
the RGB values of the image are read and a pixel is
classied as either black or white and the classication
result is stored for further re-use in the subsequent steps.
Moreover, whenever the tracking is successful, only the
relevant pixels are thresholded and processed by the two-
step 
ood ll segmentation. Clearing the per-pixel classi-
cation memory area is also efficiently performed by only
resetting the values inside the pattern's bounding box.
As a result, the detection step is not directly dependent
on the input image resolution, which provides a signi-
cant performance gain. If the tracking is not successful,
extra memory accesses resulting from this on-demand
strategy are negligible compared to a full-image thresh-
olding approach.

Compensation of incorrect diameter estimation
Moreover, the pixels on the black/white bor-
der are affected by chromatic aberration, nonlinear cam-
era sensitivity, quantization noise, and image compres-
sion

The compensation of the
pattern diameter reduces the average localization error
by approximately

Moreover,
we found that the OpenCV's blob radius calculation was
too imprecise to reliably estimate the pattern distance
and could not be used for the full 3D localization.


 Yang,
Scherer, and Zell (2013) used an H‐shaped target that could be
recognized successfully in complex scenes and had better robustness.

Ellipse extraction
Ellipses are the most significant features of the target and can give the
center of the target. Fornaciari, Prati, and Cucchiara (2014) provided a
novel real‐time algorithm called yet another ellipse detector (YAED)
relying on an innovative arc selection strategy. YAED segments contour
into four quadrants. The arcs that can group an ellipse are selected in
these quadrants. Figure 5e shows the ellipses detected by YAED.

\bibliography{gdop} 
\bibliographystyle{ieeetr}

\end{document}